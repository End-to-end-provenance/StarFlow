\documentclass[11pt]{article} 

\oddsidemargin 0.0in 
\evensidemargin
\oddsidemargin 
\setlength{\textwidth}{6.5in} 
\setlength{\textheight}{9.0in} 
\setlength{\topmargin}{-0.5in} 
\setlength{\headheight}{0.0in} 
\setlength{\headsep}{0.0in}
\setlength{\headheight}{0.0in}
\setlength{\headsep}{0.5in}

\newcommand{\noi}{\noindent}


\begin{document} 

\title{StarFlow To Do}
\maketitle

\section{Top-priority things to do}

\subsection{Testing}
Errors in the graph propagation algorithms. Carefully designed unit testing and scenario testing. 

\subsection{Extend the design of the LinkList}
Independent of how the LinkList is stored, its design needs to be extended to support runtime 
analysis links, \verb|Call()| links, links for different languages.  

\begin{itemize}

	\item{Need to store information about what language a script is written in (or maybe this is 
	just inferred by file extension, and possibly also what is called to execute the script, e.g.
	\verb|PYTHON_PATH|).}
	
	\item{Need to store parameters passed to a function, e.g. during runtime.  This relates to how
	parameters are passed to functions in protocol instances, and should be handled the same way.
	(Aside:  this doesn't work for parameters that are functions, and I have possibly seen this
	fail for nontrivial data structures like a 2d \verb|numpy.ndarray|.  What are our limitations?)}
	
	\item{Possibly store the actual command that is executed.  This would be redundant information
	but possibly useful.}
	
\end{itemize}

\subsection{Refactor so Python-specific functions are modular wrt the rest of the code}

\begin{itemize}

	\item{\textbf{LinkManagement.py} Separate Python-specific link extraction, module finding, etc. 
	from graph propagation operations.}
	
	\item{\textbf{Update.py} Parameterize language-specific information (e.g. \verb|PYTHON_PATH|).}

	\item{\textbf{Storage.py} Modularize general file storage with respect to Python-specific module 
	storage. In the general case probably would want to have different places for StoredModules for 
	different languages?}

	\item{\textbf{StaticAnalysis.py} Obviously, these routines are for Python scripts, and belong in 
	a place for Python-specific stuff.  My understanding is that these functions meet our needs for 
	static analysis, and could be a useful stand-alone package for others to use.  Should we make 
	this a little package?  Figure out if changes should be made to the API to be more generally 
	useful.}

	\item{\textbf{MetaData.py} Metadata collection from Python returns during runtime should be 
	separated from general metadata management.  Our management of metadata (for files and Python 
	scripts) should function as, and perhaps be offered as, a stand-alone package (with a GUI
	browser/API).}
	
	\item{\verb|system_io_override.py| is Python-specific and does stuff that would be leveraged
	for extracting dependencies during runtime analysis.}

\end{itemize}

\subsection{Optimize propagation algorithms, possibly extend the graph querying API}

\begin{itemize}

	\item{Investigate graph databases as a solution for storing and querying the LinkList, in 
	particular caching for graph propagation algorithms. Figure out how this would connect to 
	runtime analysis and protocols.  Implementing a graph database solution would optimally allow
	us to rewrite graph propagation and querying in }
	
	\item{How do mercurial, git store revision history?  Connecting to a version control system
	could mean querying and using its provenance, or directly extending its provenance model.}
	
	\item{Think about if we should add algorithms for additional queries.  If we implement a graph
	database solution, this could be done by exposing its query language in our API.}
	
\end{itemize}

\subsection{Dependency extraction during runtime analysis; Call() and local pickling}

Extend \verb|system_io_override.py| to add dependencies extracted during runtime to the LinkList.
If it makes sense, we should take advantage of work by IncPy. 


\subsection{Object-oriented support for protocols}

First do the \verb|@protocol| decorator.  
Use \verb|inspect.getmodule|, and then probably use \verb|__import__|.

\noi It would be great if this had a better API.  Some potential ideas:  
An abstract protocol is an instance of a the \verb|protocol| class.  The abstract steps in the 
protocol should be available somehow, e.g. as attributes of a protocol.  Protocol instances are
instantiated from an abstract protocol, and these are some other data object, e.g. the 
\verb|pipeline| class? It would be great if this was an in-memory object with some decent 
functionality that let you easily look within and across specific pipeline instances.
By looking at the \verb|@protocol| decorators when you create a pipeline instance, this object 
could be produced with \verb|depends_on| and \verb|creates| annotations.  It could basically be a 
LinkList for the protocol.  


\subsection{Sphinx documentation}
Necessary if we are giving a talk / doing a release.
	

\section{Other stuff to do}

\subsection{Connection to version control, and connection to data archiving}

The connection to version control is interesting, but we should evaluate whether we think it is
worth our time.  We obviously want to play well with version control, and we may want to think about
how to unify an existing version control system with a data storage solution.  Operations like
DE extraction / integration could be integrated with version control ideas like forking / merging.
Think about LinkList and code revision history.  Also this is interesting to PASS, PANDA, etc.

\subsection{DEServe}


\section{Things that should be plug-ins (think about this more carefully)}
\begin{itemize}
	\item{Anything that depends on the ``choice of scripting language''}
	\item{Diff / hash checkers, including file-specific diff-checkers?}
	\item{Parallelization}
	\item{Metadata and browser stuff}
	\item{Version system}
	\item{One of the slowest things is LoadLiveModules, could let users define...}
\end{itemize}

\section{Other comments / questions}

\subsection{Supporting other languages}
Where is the interactive prompt?  How are protocols handled?
	
\subsection{Metadata}
Our handling of metadata is good and flexibly meets our needs.  
Further improvements / extensions will be in the form of GUI support.


\section{Stuff that Justin is doing}

\begin{itemize}
	\item{Write \verb|setup.py| [DONE]}
	\item{Rename \verb|System| to \verb|starflow| and move docs outside of \verb|starflow| [DONE]}
	\item{Place initialize in a new module, \verb|starflow.interactive|, so that the user types \\
	\verb|from starflow.interactive import *| \\
	instead of \verb|execfile(`../starflow/initialize')| [DONE]}
	\item{Liberate having to run from Temp [next week]}
	\item{Move DE-specific config files to .starflow and figure out what happens when a DE is 
	registered (\verb|PerMachineSetup.py|, \verb|configure_live_module_filters.txt|, \\
	\verb|configure_automatic_updates.txt|, \verb|SetupFunctions.py|)}
	\item{Overall .starflow for global configs and possibly registration of each DE 
	(PerMachineSetup)}
	\item{Write out cached stuff to .starflow (StoredLinks, StoredModules, Tmp, MetaData, Archive)}
	\item{Shell API; API for DE registration and registration certificate for each DE}
	\item{PEP-8}
\end{itemize}

The GUI browser / API (flesh this out).
	
\end{document}