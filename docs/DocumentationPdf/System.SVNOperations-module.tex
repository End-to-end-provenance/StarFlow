%
% API Documentation for DataEnvironment
% Module System.SVNOperations
%
% Generated by epydoc 3.0.1
% [Tue Mar 31 17:17:20 2009]
%

%%%%%%%%%%%%%%%%%%%%%%%%%%%%%%%%%%%%%%%%%%%%%%%%%%%%%%%%%%%%%%%%%%%%%%%%%%%
%%                          Module Description                           %%
%%%%%%%%%%%%%%%%%%%%%%%%%%%%%%%%%%%%%%%%%%%%%%%%%%%%%%%%%%%%%%%%%%%%%%%%%%%

    \index{System \textit{(package)}!System.SVNOperations \textit{(module)}|(}
\section{Module System.SVNOperations}

    \label{System:SVNOperations}
Routines for running the "flat" SVN archive that allows storage of files 
w/o .svn directories.

The basic idea is that:  files that the user wants to put in SVN are 
copied, with a naming convention, to a single large flat SVNArchive 
directory. Then, files in this directory are actually the ones put under 
SVN control.

This module provides convenience functions for managing this process, 
including CheckIn(), Update(), and Status().

Each of these functions has 3 basic steps:

1) before doing anything, the archive is "Equalized", meaning that -- at 
least for the directories the user has specified in a configuration file --
the achive is made to contain the exact same contents as the user's real 
directories. The user's file system is always treated with priority -- 
whatever the contents there, that is what the archive is made to reflect.

2) then the SVN action occurs, be it Check In or Update, or Status.

3) the the users' real file system is Equalized with the archive, to get 
any new or modified files or delete those that have been removed in the SVN
repository.

To use these routines the environment variable 'SVNArchivePath' should be 
set -- this reflects the location of the SVNArchive.

Each of them takes an optional SVNPath argument, which can be used in place
of the users' specified confiuration file.


%%%%%%%%%%%%%%%%%%%%%%%%%%%%%%%%%%%%%%%%%%%%%%%%%%%%%%%%%%%%%%%%%%%%%%%%%%%
%%                               Functions                               %%
%%%%%%%%%%%%%%%%%%%%%%%%%%%%%%%%%%%%%%%%%%%%%%%%%%%%%%%%%%%%%%%%%%%%%%%%%%%

  \subsection{Functions}

    \label{System:SVNOperations:CheckIn}
    \index{System \textit{(package)}!System.SVNOperations \textit{(module)}!System.SVNOperations.CheckIn \textit{(function)}}

    \vspace{0.5ex}

\hspace{.8\funcindent}\begin{boxedminipage}{\funcwidth}

    \raggedright \textbf{CheckIn}(\textit{SVNPaths}={\tt None})

    \vspace{-1.5ex}

    \rule{\textwidth}{0.5\fboxrule}
\setlength{\parskip}{2ex}
    Run checkin on the the "flat" SVN archive from user specified 
    directories.

    ARGUMENT: --SVNPaths = list of directory paths to look in.  Normally 
    this argumnt is not given, and instead the files are generated by a 
    user-specified configuration file. See commens for the function 
    "ProcessSVNPaths"

\setlength{\parskip}{1ex}
    \end{boxedminipage}

    \label{System:SVNOperations:Update}
    \index{System \textit{(package)}!System.SVNOperations \textit{(module)}!System.SVNOperations.Update \textit{(function)}}

    \vspace{0.5ex}

\hspace{.8\funcindent}\begin{boxedminipage}{\funcwidth}

    \raggedright \textbf{Update}(\textit{SVNPaths}={\tt None})

    \vspace{-1.5ex}

    \rule{\textwidth}{0.5\fboxrule}
\setlength{\parskip}{2ex}
    Run update on the the "flat" SVN archive from user specified 
    directories.

    ARGUMENT: --SVNPaths = list of directory paths to look in.  Normally 
    this argumnt is not given, and instead the files are generated by a 
    user-specified configuration file. See commens for the function 
    "ProcessSVNPaths"

\setlength{\parskip}{1ex}
    \end{boxedminipage}

    \label{System:SVNOperations:Status}
    \index{System \textit{(package)}!System.SVNOperations \textit{(module)}!System.SVNOperations.Status \textit{(function)}}

    \vspace{0.5ex}

\hspace{.8\funcindent}\begin{boxedminipage}{\funcwidth}

    \raggedright \textbf{Status}(\textit{SVNPaths}={\tt None})

    \vspace{-1.5ex}

    \rule{\textwidth}{0.5\fboxrule}
\setlength{\parskip}{2ex}
    Run svn status on the the "flat" SVN archive from user specified 
    directories.

    ARGUMENT: -- SVNPaths = list of directory paths to look in.  Normally 
    this argumnt is not given, and instead the files are generated by a 
    user-specified configuration file. See comments for the function 
    "ProcessSVNPaths"

\setlength{\parskip}{1ex}
    \end{boxedminipage}

    \label{System:SVNOperations:EqualizeArchive}
    \index{System \textit{(package)}!System.SVNOperations \textit{(module)}!System.SVNOperations.EqualizeArchive \textit{(function)}}

    \vspace{0.5ex}

\hspace{.8\funcindent}\begin{boxedminipage}{\funcwidth}

    \raggedright \textbf{EqualizeArchive}(\textit{Files}, \textit{SVNPaths}, \textit{ArchivePath}={\tt None})

    \vspace{-1.5ex}

    \rule{\textwidth}{0.5\fboxrule}
\setlength{\parskip}{2ex}
    Equalizes archive to reflect exactly identical contents to the users' 
    filesystem outside the archive.

\setlength{\parskip}{1ex}
    \end{boxedminipage}

    \label{System:SVNOperations:EqualizeRealFiles}
    \index{System \textit{(package)}!System.SVNOperations \textit{(module)}!System.SVNOperations.EqualizeRealFiles \textit{(function)}}

    \vspace{0.5ex}

\hspace{.8\funcindent}\begin{boxedminipage}{\funcwidth}

    \raggedright \textbf{EqualizeRealFiles}(\textit{Files}, \textit{SVNPaths}, \textit{ArchivePath})

    \vspace{-1.5ex}

    \rule{\textwidth}{0.5\fboxrule}
\setlength{\parskip}{2ex}
    Equalizes users real filesystem to reflect contents of SVN archive.

\setlength{\parskip}{1ex}
    \end{boxedminipage}

    \label{System:SVNOperations:ProcessSVNPaths}
    \index{System \textit{(package)}!System.SVNOperations \textit{(module)}!System.SVNOperations.ProcessSVNPaths \textit{(function)}}

    \vspace{0.5ex}

\hspace{.8\funcindent}\begin{boxedminipage}{\funcwidth}

    \raggedright \textbf{ProcessSVNPaths}(\textit{SVNPaths})

    \vspace{-1.5ex}

    \rule{\textwidth}{0.5\fboxrule}
\setlength{\parskip}{2ex}
    Given list of paths, returns a list of files in these paths to copy to 
    and from the SVN archive.

    ARGUMENT: -- SVNPaths = a list of paths, or None

    If SVNPaths = None, the function looks for an enviroment variable 
    called PathToSVNInfo, which specifies the path to a user-specified file
    containing a list of directories to use as SVN paths.  (One directory 
    per line)

    Once SVNPaths is determined, the files in those directories are 
    determined. This can either be done by the default, which is simply to 
    get recursively all files in teh directories, or by some user-specified
    function.  To make the latter work, you specify the environment 
    variable 'DotPathToSVNFilterFunction' which should be a python dot path
    to a python function. This function must take the argument SVNPaths 
    (which may be None), and must return [Files,SVNPaths].

\setlength{\parskip}{1ex}
    \end{boxedminipage}

    \label{System:SVNOperations:AddAll}
    \index{System \textit{(package)}!System.SVNOperations \textit{(module)}!System.SVNOperations.AddAll \textit{(function)}}

    \vspace{0.5ex}

\hspace{.8\funcindent}\begin{boxedminipage}{\funcwidth}

    \raggedright \textbf{AddAll}(\textit{ArchivePath}={\tt None})

    \vspace{-1.5ex}

    \rule{\textwidth}{0.5\fboxrule}
\setlength{\parskip}{2ex}
    calls a simple command line awk script that svn adds all non-added 
    files in a directory

\setlength{\parskip}{1ex}
    \end{boxedminipage}

    \label{System:SVNOperations:DeleteMissing}
    \index{System \textit{(package)}!System.SVNOperations \textit{(module)}!System.SVNOperations.DeleteMissing \textit{(function)}}

    \vspace{0.5ex}

\hspace{.8\funcindent}\begin{boxedminipage}{\funcwidth}

    \raggedright \textbf{DeleteMissing}(\textit{ArchivePath})

    \vspace{-1.5ex}

    \rule{\textwidth}{0.5\fboxrule}
\setlength{\parskip}{2ex}
    calls a simple command line awk script to svn del all missing fles in a
    directory USE WITH CARE!!!!

\setlength{\parskip}{1ex}
    \end{boxedminipage}

    \label{System:SVNOperations:RealName}
    \index{System \textit{(package)}!System.SVNOperations \textit{(module)}!System.SVNOperations.RealName \textit{(function)}}

    \vspace{0.5ex}

\hspace{.8\funcindent}\begin{boxedminipage}{\funcwidth}

    \raggedright \textbf{RealName}(\textit{apath}, \textit{ArchivePath}={\tt None})

    \vspace{-1.5ex}

    \rule{\textwidth}{0.5\fboxrule}
\setlength{\parskip}{2ex}
    converts an archive file path to the corresponding path that the 'real 
    file' would have

\setlength{\parskip}{1ex}
    \end{boxedminipage}

    \label{System:SVNOperations:SVNArchiveName}
    \index{System \textit{(package)}!System.SVNOperations \textit{(module)}!System.SVNOperations.SVNArchiveName \textit{(function)}}

    \vspace{0.5ex}

\hspace{.8\funcindent}\begin{boxedminipage}{\funcwidth}

    \raggedright \textbf{SVNArchiveName}(\textit{path}, \textit{ArchivePath}={\tt None})

    \vspace{-1.5ex}

    \rule{\textwidth}{0.5\fboxrule}
\setlength{\parskip}{2ex}
    converts a 'real file" path to the corresponding path that copy in the 
    SVN archive should have

\setlength{\parskip}{1ex}
    \end{boxedminipage}


%%%%%%%%%%%%%%%%%%%%%%%%%%%%%%%%%%%%%%%%%%%%%%%%%%%%%%%%%%%%%%%%%%%%%%%%%%%
%%                               Variables                               %%
%%%%%%%%%%%%%%%%%%%%%%%%%%%%%%%%%%%%%%%%%%%%%%%%%%%%%%%%%%%%%%%%%%%%%%%%%%%

  \subsection{Variables}

    \vspace{-1cm}
\hspace{\varindent}\begin{longtable}{|p{\varnamewidth}|p{\vardescrwidth}|l}
\cline{1-2}
\cline{1-2} \centering \textbf{Name} & \centering \textbf{Description}& \\
\cline{1-2}
\endhead\cline{1-2}\multicolumn{3}{r}{\small\textit{continued on next page}}\\\endfoot\cline{1-2}
\endlastfoot\raggedright D\-e\-f\-a\-u\-l\-t\-S\-V\-N\-A\-r\-c\-h\-i\-v\-e\-P\-a\-t\-h\- & \raggedright \textbf{Value:} 
{\tt '../SVNArchiveArea/'}&\\
\cline{1-2}
\end{longtable}

    \index{System \textit{(package)}!System.SVNOperations \textit{(module)}|)}
